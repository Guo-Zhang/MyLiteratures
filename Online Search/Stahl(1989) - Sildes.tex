\documentclass{beamer}

% packages
\usepackage{hyperref}
\usepackage{lmodern} % http://ctan.org/pkg/lm 

\begin{document}

\section{Introduction}
\begin{frame}

\end{frame}


\section{Optimal Consumer Search}

\begin{frame}
\frametitle{Assumptions}
\begin{itemize}
	\item \textbf{Store price $z$ as given}
	\item Determine price distribution $F(p;z)$ of NE in consumer search conditional on z
\end{itemize}
\end{frame}

\begin{frame}
\frametitle{}
$$
CS(p;z)\equiv \int_p^z D(x)dx
$$
\begin{itemize}
	\item $z\geq0$: observed price
	\item $CS(0;z)<\infty$
\end{itemize}
\end{frame}

\begin{frame}[allowframebreaks]
\frametitle{}
$$
ECS(z)\equiv \int_{b}^{z}CS(p;z)dF(p)
$$
\begin{itemize}
	\item F(p): price distribution
	\item b: lower bound of F(p) 
	\item P: upper bound of F(p) 
\end{itemize}
\framebreak
$$
ECS(z)\equiv \int_{b}^{z}D(p)F(p)dp
$$
\begin{itemize}
	\item Strictly increasing for all $z\in [b,P)$ as long as $D(z)>0$
\end{itemize}
\end{frame}

\begin{frame}
\frametitle{Symmetric NE Condition}
\begin{itemize}
	\item Sysmetric search cost $c$
	\item Symmetric NE condition: $ECS(z)=c$
	\item Solution: $z=r_F$
	\begin{itemize}
		\item $r_F$ exist
		\item $r_F=\infty$ otherwise
	\end{itemize}
	\item Implication: when $p<r_F$, Consumers stop searching
\end{itemize} 
\end{frame}

\begin{frame}
\frametitle{}
\begin{itemize}
	\item $c>0$
	\item $c=0$(shoppers): all consumers will visit the store with the lowerest price only
\end{itemize}
\end{frame}


\section{Store Price-Setting}
\begin{frame}
\frametitle{Assumptions}
\begin{itemize}
	\item \textbf{Consumer reservation price $r$ as given}
	\item Determine F(p;r) of NE in price-setting conditional on r
	\item Number of stores, $N$, as exogenously fixed
\end{itemize}
\end{frame}

\begin{frame}
\frametitle{Lamma 1}
\begin{itemize}
	\item Given $\mu \in (0,1)$, if $F(p;r)$ is an NE-distribution conditional on reservation price $r$, then it is atomless\footnotemark
	\item Intitution: profits could be discretely increased by undercutting atoms $\to$ price distributions with atoms cannot be optimal ???
\end{itemize}
\footnotetext{\href{https://en.wikipedia.org/wiki/Atom_(measure_theory)}{Atomless?}}
\end{frame}

\begin{frame}
\frametitle{Lamma 2}
\begin{itemize}
	\item If $F(p;r)$ is an NE-distribution conditional on reservation price $r$, then $P_r = min\{r,\hat{p}\}$ ??
	\begin{itemize}
		\item $P_r$: maximal element of the support of $F(p;r)$ ??
		\item $b(r)$: minimal element of the support ??
	\end{itemize}
\end{itemize}
\end{frame}

\begin{frame}
\frametitle{}
$$
E\pi_j(p_j,F)\equiv\{\mu[1-F(p_j;r)]^{N-1}+(1-\mu)/N\}\cdot R(p_j)
$$
\begin{itemize}
	\item $E\pi_j$: profit that store j can expect when it sets price $p_j<\hat{p}$ while all other stores play $F(p;r)$
	\item First component: shoppers; probability that $p_j$ is lower than the other N-1 prices
	\item Second component: captured consumers with positive search costs
	\item $R(p_j)$: ??
\end{itemize} 
\end{frame}

\begin{frame}
\frametitle{Definition}
\begin{itemize}
	\item A distributuon $F(p;r)$ is a conditional NE if $E\pi(p,F)$ is equal to a constant($\pi$) for all $p$ in the support of $F$ and not greater than $\pi$ for any $p$. $F(p;r)$ is a consistent NE if F is a conditional NE and $r$ is a consistent reservation price. ??
\end{itemize}
\end{frame}

\begin{frame}
\frametitle{}
\begin{itemize}
	\item Conditional NE-expected profit
	$$
	\pi = E\pi(P_r,F)=R(P_r)\cdot(1-\mu)/N
	$$
	\item Solving $E\pi[p,F(p;r)]=\pi$ for $F(p;r)$
	$$
	F(p;r)=1-[(\frac{1-\mu}{N\mu})(\frac{R(P_r)}{R(p)}-1)]^{\frac{1}{N-1}}
	$$
	\item Density function:
	$$
	f(p;r) = \frac{}{} \times [(\frac{1-\mu}{N\mu})(\frac{R(P_r)}{R(p)}-1)]^{-\frac{N-2}{N-1}} \times \frac{R(P_r)}{R(p)} \frac{R'(p)}{R(p)}
	$$
\end{itemize}
\end{frame}

\begin{frame}
\frametitle{}
\begin{itemize}
	\item 
\end{itemize}
\end{frame}

\section{Asymptotic Results}
\begin{frame}
\begin{itemize}
	\item How the parameter $(\mu, N, c)$ affect
	\begin{itemize} 
	\item Consistent reservation price: $\rho(\mu,N,c)$
	\item Resulting NE distribution of prices: $\Phi(p;\mu,N,c)\equiv F[p;\rho(\mu,N,c),\mu,N]$
	\item Lower bound of the support of this distribution: $\beta(\mu,N,c)\equiv b[\rho(\mu,N,c),\mu,N]$
	\item Upper bound of the support of this distribution: $P_r(\mu,N,c)\equiv min\{\rho(\mu,N,c),\hat{p}\}$
	\item NE-expected profits: $\pi(\mu,N,c)\equiv[P_r(\mu,N,c)]\cdot(1-\mu)/N$
	\end{itemize}
\end{itemize}
\end{frame}

\begin{frame}
\begin{itemize}
	\item $\mu$ approaches 0 or 1
	\item $c$ approaches 0 or increases without bound
	\item $N$ increases without bound
\end{frame}

\section{Local Comparative Statics}

\section{Welfare Analysis}

\section{Discussion}
\begin{frame}
\begin{itemize}
	\item Bridge between Bertrand and Diamond results
	\begin{itemize}
		\item As the proportion of shoppers goes from 1 to 0, the NE changes continuously from the Bertrand NE to the Diamond NE.
		\item As the cost of search goes to zero, the NE converges to the Bertrand NE.
		\item As the number of stores increase, the NE moves toward the Diamond NE.
		\begin{itemize}
			\item With more stores, the probability of being the lowest-priced store descreases exponentially.
			\item $\to$ Enter discourages more competitive pricing.
		\end{itemize}
		\item Welfare effects
		\begin{itemize}
		     \item Shoppers benefit from having more stores.
		     \item Effects on total social welfare depend on demand specifications.
		     \item Total Surplus declines monotonically with entry.
	    \end{itemize}
	\end{itemize}
\end{itemize}
\end{frame}

\begin{frame}
\begin{itemize}
	\item 
\end{itemize}
\end{frame}

\begin{frame}
\begin{itemize}
	\item Permit more general distributions of search costs
	\item Permit a postitive cost to second visits
	\begin{itemize}
		\item Perfect recall but return costly
		\item $\to$Alter the reservation prices of cousumers
	\end{itemize}
	\item Asymmetric NE
\end{itemize} 
\end{frame}

\begin{frame}
\begin{itemize}
	\item Increase marginal costs
	\item Fixed costs
	\item Capacity constraints
	\item Heterogeneous marginal costs
\end{itemize}
\end{frame}

\begin{frame}
\frametitle{}
\begin{itemize}
	\item Consumers act over time
	\item Stores adopt static strategies
	\end{itemize}
\end{frame}

\end{document}
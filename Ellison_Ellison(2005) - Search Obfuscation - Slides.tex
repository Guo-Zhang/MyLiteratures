\documentclass{beamer}

% theme
\mode<presentation>
{
\usetheme{Warsaw}

\setbeamercovered{transparent}
}

% packages
\usepackage{multicol}
\usepackage{amsmath}

% settings
\linespread{1.2}
\setbeamertemplate{frametitle continuation}{}

% title page
\title{Search, Obfuscation, and Price Elasticities on the Internet}
\subtitle{Glenn Ellison and Sara Fisher Ellison, 2009}
\author{Guo Zhang}
\institute[Universities of]
{
WISE, Xiamen University
}
\date{This Version: \today}
\subject{Literatures}


% body
\begin{document}

% title page
\begin{frame}[plain]
\maketitle
\end{frame}

% content
\begin{frame}[plain] %[allowframebreaks] 
\frametitle{Contents}
\small
\begin{multicols}{2}
  \tableofcontents
\end{multicols}
% You might wish to add the option [pausesections]
\end{frame}

\section{Introduction}
% understand
\begin{frame}
\frametitle{Motivation}
% understand
\begin{itemize}
\item Background:
  \begin{itemize}
  \item Search technology would have a dramatic effect by making it easy for consumers to compare prices at online and offline merchants.
  \item Advances in search technology are accompanied by investment by firm in obfuscation.
  \end{itemize}
\end{itemize}
\end{frame}

\begin{frame}[allowframebreaks]
\frametitle{Overview}
\begin{itemize}
\item Revelent Theory
  \begin{itemize}
  \item Obfuscation can raise search costs, leading to less consumer learning
    and higher profits.
  \item Sales of "add-ons" at high unadvertised
prices can raise equilibrium profits in a competitive price discrimination model
  \end{itemize}
  
\item Data: Pricewatch
  \begin{itemize}
  \item Not too complicated
  \item Unusually rich data
  \item Extreme aspects of the environment
  \end{itemize}
  
\framebreak

\item Informal evidence of obfuscation
  \begin{itemize}
  \item Extended-version loss-leader strategy: offer a low-quality product at a low price to attract consumers and then try to convince them
to pay more for a superior product
    \begin{itemize}
    \item Upgrade rather than buy both
    \item Loss leader may be sold for a slight profit rather than at a loss
    \end{itemize}
  \end{itemize}
  
\framebreak
  
\item Formal Empirical Analysis
  \begin{itemize}
  \item Demand and substitution patterns within four categories of computer memory modules
  \item Matching Data from two sources
    \begin{itemize}
    \item Yearlong hourly price series: repeatedly conducting price searches on Pricewatch.
    \item Sales data: a single private firm that operates several computer parts websites and derives most of its sales from Pricewatch referrals
    \end{itemize}
  \end{itemize}
    
\framebreak
\item Results
    \begin{enumerate}
    \item Price search technologies
can dramatically reduce search frictions. The firm
faces a demand elasticity of \textbf{-20} or more for its lowest quality memory modules.
    \item Charging a low price for a low-quality product
      increases our retailer's sales of medium- and high-quality products.
    
\framebreak
    
    \item Evidence of the relevance of both mechanisms
      \begin{itemize}
      \item In the search-theoretic model, obfuscation raises profits by making consumers less informed (search costs)
      \item In Ellison's (2005) add-on pricing model, obfuscation raises profits by creating an adverse-selection effect that deters price-cutting (price discrimination)
      \end{itemize}
      
    \item Retailers' obfuscation strategies have been successful in raising markups beyond the level that would otherwise be sustainable, supported by additional cost data
        \begin{itemize}
        \item Price-cost margin($\frac{p-MC}{p}$): 3\%-6\%
        \item Markup($\frac{p-MC}{MC}$): 12\%
        \end{itemize}
    \end{enumerate}
\end{itemize}
\end{frame}

\begin{frame}[allowframebreaks]
\frametitle{Literature Review}
\begin{itemize}
\item Empirical studies on price search engines
  \begin{itemize}
  \item Brynjolfsson and Smith (2001): using a data set containing the click sequences of tens of thousands of people who conducted price searches for books on
Dealtime to estimate several discrete-choice models of demand.
  \item Baye, Gatti, Kattuman, and Morgan (2006): an extensive data set on the Kelkoo price comparison site, finding that there is a big discontinuity in
  clicks at the top, in line with clearinghouse models.
  \end{itemize}
  
\framebreak
  
\item Online price dispersion
  \begin{itemize}
  \item Price elasticities obtained from quantity data in an online retail sector: Chevalier and Goolsbee (2003).
  \item Internet search and price levels: Brown and Goolsbee (2002); Scott Morton, Zettelmeyer, and Silva-Risso (2001, 2003).
  \end{itemize}
  
\end{itemize}
\end{frame}

\begin{frame}
\frametitle{Contribution}
\begin{itemize}
  \item Actual quantities rather than clicks
  \item Quantity numbers rather than quantity ranks
  \item Spawned a broader literature on obfuscation
\end{itemize}
\end{frame}

\section{Theory of Search and Obfuscation}
% understand
\subsection{Incomplete Consumer Search}
\begin{frame}
\frametitle{Incomplete Consumer Search}
\begin{itemize}
  \item Stahl (1989,1996): a model with search cost - mixed strategy randomizing over prices with some interval; fully informed consumer purchase with lowest price and others stop searching before finding the lowest
  \item Basic intuition from search models: obfuscation might lead to higher profits by making consumer learning less complete
\end{itemize}
\end{frame}

\subsection{Add-Ons and Adverse Selection}
\begin{frame}[allowframebreaks]
\frametitle{Add-Ons and Adverse Selection: Setup}
\begin{itemize}
\item Two firm i=1,2
\item Two versions of goods j=L,H
\item Constant marginal costs $c_L$ and $c_H$;\\ upgrade cost $c_U=c_H-c_L$
\item Post prices $p_{iL}$ and nonposted prices $p_{iH}$;\\ upgrade price $p_{iU}\equiv p_{iH}-p_{iL}$
\item Time cost per website s
\item Buy at most one unit

\framebreak

\item Incremental price of the "upgrade" $\varepsilon$: for $\varepsilon<s$, no consumer will switch to the other firm
\item $x(p_{iU},p_{iL},p_{-iL})$: the fraction of consumers choosing to upgrade
\item $p*_{iU}(p_{iL},p_{-iL})=p^m_{iU}(p_{iL},p_{-iL})\equiv Arg \max_p (p-c_U)x(p,p_{iL},p_{-iL})$
\item $x*(p_{1L},p_{2L})$ for $x(p*_{iU}(p_{iL},p_{-iL}))$
\item  $D_1(p_1,p_2)$: number of consumers who visit firm 1 (In any pure strategy equilibrium, all consumers who visit firm i will buy from firm i)
\end{itemize}
\end{frame}

\begin{frame}[allowframebreaks]
\frametitle{Add-Ons and Adverse Selection: Model}
\begin{itemize}
\item Firm 1's profit:
\begin{align*}
\pi_1(p_{1L},p*_{2L})&=(\text{unit profit from low quality}\\
 &\quad +\text{fraction of upgrade}*\text{unit profit from upgrade})\\
 &\quad *\text{number of consumers buying} \\
&=[(p_{1L}-c_L)+x^*(p_{1L},p^*_{2L})*(p^m_{1U}(p_{1L},p^*_{2L})-c_U)]\\
&\quad *D_1(p_{1L},p*_{2L}) \\
\end{align*}

\framebreak
\item First-order condition:
\begin{align*}
\frac{\delta \pi_1}{\delta p_{1L}}&=\frac{\delta D_1}{\delta p_{1L}} (p_{1L}-c_L+x^*(p_{1L},p^*_{2L})(p^m_{1U}(p_{1L},p^*_{2L})-c_U))\\
&\quad +D_1(p_{1L},p^*_{2L})[1+\frac{\delta x^*}{\delta p_{1L}}(p^*_{1U}(p_{1L},p^*_{2L})-c_U)\\
&\quad +x^*(p_{1L},p^*_{2L})\frac{\delta p^m_{1U}}{\delta p_{1L}}] \\
\end{align*}

\framebreak
\begin{align*}
\intertext{Let}
\varepsilon&=\frac{\delta D_1}{\delta P_{1L}}\frac{p^*_{1L}+x^*(p_{1L},p^*_{2L})p^m_{1U}}{D_1(p^*_{1L},p^*_{2L})} \\
\intertext{Therefore,}
&\quad \frac{p^*_{1L}-c_L+x^*(p^m_{1U}-c_U)}{p^*_{1L}+x^*(p_{1L},p^*_{2L})p^m_{1U}} \\
&=-\frac{1}{\varepsilon}(1+\frac{\delta x^*}{\delta p_{1L}}(p^*_{1U}(p_{1L},p^*_{2L})-c_U)+x^*(p_{1L},p^*_{2L})\frac{\delta p^m_{1U}}{\delta p_{1L}})\\
\end{align*}

\item First-order condition:
\begin{align*}
\text{\tiny Firm's revenue-weighted average markup}&=\text{\tiny Inverse of a demand elasticity and a multiplier}
\intertext{Suppose $p_{1U}$ is independent of $p_{1L}$,}
\frac{\delta p^m_{1U}}{\delta p_{1L}}&=0 \\
\frac{\delta x*}{\delta p_{1L}}&=0 \\
\intertext{Therefore,}
\frac{p^*_{1L}-c_L+x^*(p^m_{1U}-c_U)}{p^*_{1L}+x^*(p_{1L},p^*_{2L})p^m_{1U}}&=-\frac{1}{\varepsilon}\\
\end{align*}

\end{itemize}
\end{frame}

\begin{frame}
\frametitle{Add-Ons and Adverse Selection: Implication}
\begin{itemize}
\item Constant-upgrade-fraction assumption is not compelling(Ellison, 2005)
  \begin{itemize}
  \item Price cuts disproportionately attract cheap models who have a lower willingness to pay for upgrades:
  $\frac{\delta p^m_{1U}}{\delta p_{1L}}>0$,$\frac{\delta x^*}{\delta p_{1L}}>0$
  \end{itemize}
\item Such demand systems has an adverse-selection problem when add-ons are sold.
    \begin{itemize}
    \item Sales of add-ons will raise equilibrium profit margins above the inverse-elasticity benchmark
    \item Taking a low-cost, high-value feature out of the
low-quality good and making it available in the high-quality good may be a profit-enhancing strategy.
    \end{itemize}
\end{itemize}
\end{frame}

\section{The Pricewatch Universe And Memory Modules}

\section{Observations of Obfuscation}

\section{Data}

\begin{frame}
\frametitle{Variables}
\begin{itemize}
\item LowestPrice: lowest price listed on Pricewatch
\item Range 1-12: the difference between the twelfth lowest listed price and the lowest listed price
\item PLow, PMid and PHi: prices for tree qualities of memory modules at the two websites
\item QLow, QMid, and QHi: average daily quanitities of each quality of module sold by each website
\item PLowRank: the rank of the website's first entry in Pricewatch's sorted list of prices within the category
\end{itemize}
\end{frame}

\section{Demand Patterns}
\subsection{Methodology for Demand Estimation}
\begin{frame}[allowframebreaks]
\frametitle{Methodology for Demand Estimation}
\begin{itemize}
\item Product category: c
\item Quality: q
\item Website: w
\item Day: t
\end{itemize}

\framebreak

\begin{align*}
Q_{wcqt}&=e^{X_{wct}\beta_{cq}}u_{wcqt}
\intertext{with}
X_{wct}\beta_{cq}&=\beta_{cq0}+\beta_{cq1}log(PLow_{wct})+\beta_{cq2}log(PMid_{wct})\\
&\quad+ \beta_{cq3}log(PHi_{wct})+\beta_{cq4}log(LowestPrice_{ct})\\
&\quad+\beta_{cq5}log(1+PLowRank_{wct})+\beta_{cq6}Weekend_t\\
&\quad+\beta_{cq7}SiteB_w + \sum_{s=1}^{12}\beta_{cq(7+s)}TimeTrend_{st}
\end{align*}
\end{frame}

\subsection{Basic Results on Demand}
\begin{frame}[allowframebreaks]
\frametitle{Demand for 128MB PC100 Memory Modules(Table II)}
\begin{itemize}
\item Demand for low-quality modules at a website is extremely price-sensitive - the effect of Pricewatch rank on demand
    \begin{itemize}
    \item Coefficient on log(1+PLowRank): moving from first to seventh reduces 83\% ??
    \item Highly significant
    \end{itemize}
    
\framebreak
 
\item Low-quality memory is an effective loss leader
\begin{itemize}
\item Coefficients on log(1+PLowRank) in the second and third columns are negative and highly significant - higher position, higher sales
\item Effect is strong: medium - 66\%; high - 51\%  ??
\item Pricewatch ranks change frequently, whereas medium- and high-quality prices are left unchanged for substantial periods of time, so that most of the
variation in the attractiveness of our firm's medium- and high-quality prices will occur around the occasional price changes.
\end{itemize}
\item Site B dummy are negative and significant - website design is important
\end{itemize}
\end{frame}

\begin{frame}
\frametitle{Price Elasticities for Memory Modules(Table III)}
\begin{itemize}
\item An own-price elasticity of ?24.9 for low-quality 128MB PC100 modules.
\item  low-quality products have highly elastic demand and that there are loss-leader benefits from selling low-quality goods at a low price are consistent across categories
\end{itemize}
\end{frame}

\subsection{The Mechanics of Obfuscation: Incomplete Consumer Search}
% not understand
\begin{frame}[allowframebreaks]
\frametitle{The Mechanics of Obfuscation: Incomplete Consumer Search}
\begin{itemize}
\item Motivation:
\begin{itemize}
\item An alternate explanation for the finding could be that PLowRank is correlated with the rank of a site's higher quality offerings
\end{itemize}
\item Method:
\begin{itemize}
\item Logit models
\item Dependent variable: Site A
\item Independent variable: log(1+PLowRank),log(PMid),log(PHi),time trends
\end{itemize}

\framebreak
\item Results(Table IV):
\begin{itemize}
\item Consumers are influenced by the prices of the product they are buying
\item Consumers are also more likely to purchase from the site with a lower low-quality price
\end{itemize}
\end{itemize}
\end{frame}

\subsection{The Mechanics of Obfuscation: Add-Ins and Adverse Selection}
\begin{frame}
\frametitle{The Mechanics of Obfuscation: Add-Ins and Adverse Selection}
\begin{itemize}
\item If the elasticity on the low-quality memory is larger (in absolute value) than that for medium- or high-quality memory, there is evidence of adverse selection (Note on constant-fraction assumption of Section 2)
\item  Firm's quality mix using sample means: 63\% low-quality in first place; 35\% in tenth place ??
\end{itemize}
\end{frame}

\subsection{Instrumental Variables Estimates}
% not understand
\begin{frame}
\frametitle{Instrumental Variables Estimates}
\begin{itemize}
\item 
\end{itemize}
\end{frame}

\section{Markups}
\begin{frame}
\frametitle{}
\begin{itemize}
\item 
\item 
\end{itemize}
\end{frame}


\end{document}

 
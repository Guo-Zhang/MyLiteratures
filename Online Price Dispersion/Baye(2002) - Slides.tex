\documentclass{beamer}

% theme
\mode<presentation>
{
\usetheme{Warsaw}

\setbeamercovered{transparent}
}

% packages
% \usepackage{multicol}

% setting
\linespread{1.2}

% title page
\title{Price Dispersion in the Small and in the Large: Evidence from an Internet Price Comparison Site}
\subtitle{Baye et. al.}
\author{Guo Zhang}
\institute{WISE, Xiamen University}
\date{This Version: \today}
\subject{Literatures}

% body
\begin{document}

\begin{frame}
\maketitle
\end{frame}

\begin{frame}[plain]
\frametitle{Contents}
% \begin{multicols}{2}
\tableofcontents[hideallsubsections]
% \end{multicols}
\end{frame}

\section{Introduction}
\begin{frame}
\frametitle{}
\begin{itemize}
\item 
\end{itemize}
\end{frame}

\section{Theory}
\begin{frame}
\frametitle{Measurement of Price Dispersion}
\begin{itemize}
\item Coefficient of variation: $CV=\sigma/\mu$
\begin{itemize}
\item Cannot capture the situation when Bertrand competition exists in the lowest two products
\end{itemize}
\item The gap: $G=p_2-p_1$
\end{itemize}
\end{frame}

\begin{frame}
\frametitle{Convergence Hypothesis}
\begin{itemize}
\item While price dispersion may be positive at an instant in time, the level of price dispersion (measured by G) decreases over time as Internet markets mature.
\end{itemize}
\end{frame}

\begin{frame}
\frametitle{Persistence Hypothesis}
\begin{itemize}
\item Price dispersion persists over time and depends systematically on the number of firms listing prices for that product. 
\item More specifically, price dispersion (measured by the Gap between the two lowest prices for a given product) is greater in the small than in the large.
\end{itemize}
\end{frame}

\begin{frame}
\frametitle{Reasons for Persistence Hypothesis}
\begin{itemize}
\item Positive search costs for comsumers
\item Private margin costs for sellers
\item Limited search for part of consumers
\end{itemize}
\end{frame}

\section{Data}
\begin{frame}
\frametitle{Data Source: Shopper.com}
\begin{itemize}
\item Price comparisons for identical consumer electronics products sold by different firms
\item Most popular 1,000 products listed 
\item Product rank - relative popularity
\item Updated twice each day, sample as well as rank changes
\end{itemize}
\end{frame}

\begin{frame}
\frametitle{Summary Statistics}
\begin{itemize}
\item 
\end{itemize}
\end{frame}

\section{Results}
\begin{frame}[allowframebreaks]
\frametitle{Results from Graphics}
\begin{itemize}
\item Based on the data for multi-price listings
\item No discernible(clear) trend on the average percentage range, average coefficient of variation and the average percentage gap.
\item Pervasive and stable phenomenon on the fraction of products for which the percentage gap exceeds 0, 1, 5, and 10 percent
\item Significantly strictly positive gap between the lowest two prices
\framebreak
\item The average percentage gap declines sharply as the number of firms listing prices increases
\item The range is significantly higher when many firms list prices than when few firms list prices
\item Difficult to discern whether the slight decline observed in the average range of prices stems from the convergence or persistence hypotheses
\end{itemize}
\end{frame}

\begin{frame}
\frametitle{Results from Econometric Model}
\begin{itemize}
\item Regress price dispersion for a particular product date against a number of dummy variables that capture the effects of differences in market structure across products and across time
\end{itemize}
\end{frame}



\begin{frame}[plain]
\frametitle{Contents}
% \begin{multicols}{2}
\tableofcontents[hideallsubsections]
% \end{multicols}
\end{frame}

\end{document}
%% LaTeX Beamer presentation template (requires beamer package)
%% see http://bitbucket.org/rivanvx/beamer/wiki/Home
%% idea contributed by H. Turgut Uyar
%% template based on a template by Till Tantau
%% this template is still evolving - it might differ in future releases!

\documentclass{beamer}

\mode<presentation>
{
\usetheme{Warsaw}

\setbeamercovered{transparent}
}

\usepackage[english]{babel}
\usepackage[latin1]{inputenc}

% font definitions, try \usepackage{ae} instead of the following
% three lines if you don't like this look
\usepackage{mathptmx}
\usepackage[scaled=.90]{helvet}
\usepackage{courier}


\usepackage[T1]{fontenc}

\usepackage{multicol}

\usepackage{bm}

\usepackage{amsmath}

\setbeamertemplate{frametitle continuation}{}

\title{The Dynamics of Car Sales: \\ A Discrete Choice
Approach}

\subtitle{Adda and Cooper, 2006}

% - Use the \inst{?} command only if the authors have different
%   affiliation.
%\author{F.~Author\inst{1} \and S.~Another\inst{2}}
\author{Guo Zhang}

% - Use the \inst command only if there are several affiliations.
% - Keep it simple, no one is interested in your street address.
\institute[Universities of]
{
WISE, Xiamen University
}

\date{\today}


% This is only inserted into the PDF information catalog. Can be left
% out.
\subject{Literatures}



% If you have a file called "university-logo-filename.xxx", where xxx
% is a graphic format that can be processed by latex or pdflatex,
% resp., then you can add a logo as follows:

% \pgfdeclareimage[height=0.5cm]{university-logo}{university-logo-filename}
% \logo{\pgfuseimage{university-logo}}



% Delete this, if you do not want the table of contents to pop up at
% the beginning of each subsection:
\AtBeginSection[]
{
\begin{frame}<beamer>[plain]
\frametitle{Contents}
\begin{multicols}{2}
\tableofcontents[currentsection,currentsubsection]
\end{multicols}
\end{frame}
}

\AtBeginSubsection[]
{
\begin{frame}<beamer>[plain]
\frametitle{Contents}
\begin{multicols}{2}
\tableofcontents[currentsection,currentsubsection]
\end{multicols}
\end{frame}
}



% If you wish to uncover everything in a step-wise fashion, uncomment
% the following command:

%\beamerdefaultoverlayspecification{<+->}

\begin{document}

\begin{frame}[plain]
\titlepage
\end{frame}

\begin{frame}[plain] %[allowframebreaks] %[plain]
\frametitle{Contents}
\begin{multicols}{2}
  \tableofcontents
\end{multicols}
% You might wish to add the option [pausesections]
\end{frame}

\section{Introduction}
\begin{frame}
\frametitle{Theme}
% understand
\begin{itemize}
\item Behavior of household durable consumption expenditures over time
\end{itemize}
\end{frame}

\begin{frame}
\frametitle{Motivation: Aggregate Perspective(dynamic)}
% understand
\begin{itemize}
\item Mankiw puzzle(Mankiw 1982): permanent income hypothesis(PIH) is inconsistent with observed data
    \begin{itemize}
    \item Theory: ARMA(1,1) process
    \item Empirical results: AR(1) process; depreciation rate is 100\%.
    \end{itemize}
\end{itemize}
\end{frame}

\begin{frame}
\frametitle{Motivation: Household's Perspective(heterogeneous, discrete)}
% understand
\begin{itemize}
\item A model of \textbf{heterogeneity} and \textbf{discrete adjustment} can qualitatively match relevant parts of the data.
    \begin{itemize}
    \item Lam(1991): households only \textbf{occasionally adjust} their stock of durables
    \item Bar-Ilan and Blinder (1988,1992), Bertola and Caballero (1990) and Caballero (1990,1993): view aggregate observations on durable purchases as the outcome of the aggregation over \textbf{heterogeneous} microeconomic agents
    \end{itemize}
\end{itemize}
\end{frame}

\begin{frame}[allowframebreaks]
\frametitle{Overview}
% understand
\begin{itemize}
\item Framework: Determinants of the \textbf{time series representation} of durable expenditures in an explicit \textbf{dynamic}, \textbf{discrete choice} framework
    \begin{itemize}
    \item ARMA(1,1) underlies the "Mankiw puzzle"
    \item VAR of sales, price and income - impulse reponse function
    \end{itemize}
\item Goals:
  \begin{itemize}
  \item Confronting the Mankiw puzzle for car sales
  \item Whether an aggregated discrete choice model can match and explain this rich time response to an income shock
  \end{itemize}

\framebreak
% not understand
\item Model:
  \begin{itemize}
  \item Basis: Adda and Cooper(2000a)
  \item Difference: Drawn directly form the dynamic optimization problem without imposing any structure directly on agents' decision rules(specify (S,s) bands or "desired stock" directly) % not understand
  \item Reasons:
    \begin{itemize}
    \item PIH assumptions underling "desired stock" approach are not supported by data
    \item More consistent theoretically % not understand
    \end{itemize}
  \end{itemize}

\framebreak
% not understand
\item Findings:
  \begin{itemize}
  \item
  \end{itemize}
\framebreak
% not understand
\item Sources of these dynamics:
  \begin{itemize}
  \item % not understand
  \item Sources of these dynamics
    \begin{itemize}
    \item Fluctuations or shocks in the replacement probability - most important
    \item Evolution of the cross sectional distribution of car vintages - surprisingly little
    \end{itemize}
  \end{itemize}
\end{itemize}
\end{frame}


\section{Evidence on Aggregate Car Purchases}
\begin{frame}
\frametitle{Outline}
% summarized
\begin{itemize}
\item Show the raw data on sales and cross sectional distribution over sample period
\item Test the ARMA(1,1) representation again
\item Impulse response functions from VAR on car sales, income and prices
 \begin{itemize}
 \item Illustrate why ARMA(1,1) is inadequate
 \item Evaluate the time series implication of estimated model
 \end{itemize}
\end{itemize}
\end{frame}

\subsection{Facts: Car Sales and the Cross Sectional Distribution}

\begin{frame}
\frametitle{Car Sales(Figure 1)}
% understand
\begin{itemize}
\item Measured as registrations of new cars
\item Considerable volatility
\end{itemize}
\end{frame}

\begin{frame}
% summarized
\frametitle{Cross Sectional Distribution(Figure 2)}
Pattern:
\begin{itemize}
\item New -> Old(ripple) -> Scrapped or destroyed
\item Echo effects: burst of sales -> bulge in the CDF; tempered by scrapping at earlier ages % not understand
\end{itemize}
Usage:
\begin{itemize}
\item Match moments from the CDF in the estimation of parameters % not understand
\item Variations in the CDF plan a role in explaining time series variation in sales
\end{itemize}
\end{frame}

\subsection{Time Series Representations}
\subsubsection{ARMA(1,1) Representation}
% understand
\begin{frame}
\frametitle{Extended Permanent income hypothesis model for durability}
\begin{itemize}
\item A durable good: expenditure - $e_t$; depreciation - $\delta$
\item Uncertain income: innovation to income?? - $\varepsilon_t$
\item Quadratic utility function
\end{itemize}
$$
e_{t+1}=\delta \alpha_0 +\alpha_1 e_t + \varepsilon_{t+1} - (1-\delta)\varepsilon_t
$$
\end{frame}

\begin{frame}
\frametitle{Estimation Results}
\begin{itemize}
\item Hypothesis that the rate of depreciation is close to 100\% per year would not be rejected for most of the
specifications
\item robust across
\begin{itemize}
\item Categories of durables
\item Countries
\item Time periods
\item Detrending method % not understand
\end{itemize}
\end{itemize}
\end{frame}

\subsubsection{Impulse Response Functions(IRF)}
% summarized
\begin{frame}
\frametitle{Impulse Response Functions}
\begin{itemize}
\item VAR model:
  \begin{itemize}
  \item Reason: joint dynamics of durables, income and prices over time
  \item Variables: automobile sales, automobile prices relative to the CPI,
income
  \item Order: income, prices, sales(innovations to income are exogenous, prices
respond to both price and income innovations and sales respond to innovations in all three variables) % not understand
  \item Imposed on actual data as well as the simulated data % not understand
  \item No structural interpretation
  \end{itemize}
\item Empirical results(Figure 3, P31):
  \begin{itemize}
  \item Income on sales: dampened oscillation around the baseline
    \begin{itemize}
    \item Endogenous evolution of the stock of cars can potentially produce replacement cycles
    \item Income and prices are serially correlated and have some cross dynamics
    \end{itemize}
  \item Price on sales: differs across two contries
    \begin{itemize}
    \item US: reduce
    \item France: increase
    \end{itemize}
  \end{itemize}
\end{itemize}
\end{frame}

\subsubsection{Can the ARMA model match the IRFs?}
\begin{frame}
\frametitle{Can the ARMA model match the IRFs?}
% summarized
\begin{itemize}
\item ARMA(1,1) cannot reproduce the oscillations % not understand
\item ARMA(1,1) model is structurally unable to deliver
a "depreciation rate" low enough to be credible - Mankiw
puzzle
\end{itemize}
\end{frame}

\section{Dynamic Discrete Choice Model}
\subsection{Household Behavior}
\begin{frame}
\frametitle{Starting Point}
% understand
\begin{itemize}
\item An agent with a car of age i=0,1,....
\item State(z,Z):
\begin{itemize}
\item z: vector of household specific taste \textbf{shocks}
\item $Z\equiv(p,Y,\varepsilon)$: vector of aggregate state variables
\item p: relative price of the durable good
\item Y: aggregate income
\item $\varepsilon$: \aggregate taste shock
\end{itemize}
\end{itemize}
\end{frame}

\begin{frame}
\frametitle{Household Decision}
% summarized
\begin{itemize}
\item Decision: whether to retain a car of age i or scrap it
\begin{itemize}
\item Scrap: receive the scrap value of $\pi$; purchase a new car.
\item Retain: receive the flow of services; cannot purchase another car by assumption
\item Choices influenced by a choice specific i.i.d shock $z_j$,j=k,r
\item Constant utility gain,$\alpha_k$, from keeping the car % not understand
\end{itemize}
\end{itemize}
\end{frame}

\begin{frame}
% understand
\frametitle{Initial Restrictions}
\begin{itemize}
\item No second-hand market
\item No borrowing or lending
\end{itemize}
\end{frame}

\begin{frame}[allowframebreaks]
\frametitle{Formal Model}
% summarized
\begin{itemize}
\item $V_i(z,Z)$: value of having a car of age i to a household
\item $V_i^k(z,Z)$ and $V_i^k(z,Z)$: values from keeping and scrapping an age i car in state (z,Z)
\item $\delta$: probability of car destroyed
\item $p'-\pi$: Cost of a new car
\item scrap value independent of replacement value
\end{itemize}
\begin{align}
V_i(z,Z) &= max[V_i^k(Z)+\alpha_k+z_k,V^r(Z)+z_r]
\intertext{where}
V_i^k &= u(s_i,Y,\varepsilon)+\beta(1-\delta)E_{(Z',z|Z,z)}V_{i+1}(z',Z')+\beta\delta E_{Z'|Z}V^r(Z')
\intertext{and}
V^r &= u(s_1,Y-p+\pi,\varepsilon)+\beta(1-\delta)E_{(Z',z|Z,z)}V_{2}(z',Z')+\beta\delta E_{Z'|Z}V^r(Z')
\end{align}
\framebreak
Utility function separable between durables and nodurables:
\begin{align}
u(s_i,c)=[i^{-\gamma}+\varepsilon\frac{(c/\lambda)^{1-\xi}}{1-\xi}]
\end{align}
% i? taste shock?
\begin{itemize}
\item c: consumption of non-durable goods
\item $\gamma$: curvature for the service flow of car ownership
\item $\xi$: curvature for consumption
\item $\lambda$: scale factor
\item Taste shock $\varepsilon$ influences the contemporaneous marginal rate of substitution between car services and nondurables
\end{itemize}
\framebreak
Specify the stochastic process for income, prices and the aggregate taste shocks:
\begin{align*}
Y_t &= \mu_y + \rho_{YY}Y_{t-1}+\rho_{Yp}p_{t-1}+u_{Yt} \\
p_t &= \mu_p + \rho_{pY}_{t-1} + \rho_{pp}p_{t-1}+u_{pt} \\
\varepsilon_t &= \mu_\varepsilon + \rho_{\varepsilon Y}Y_{t-1}+ \rho_{\varepsilon p}p_{t-1}+u_{\varepsilon t} \\
\end{align*}
Covariance matrix of the innovations $u=\{u_{Yt},u_{pt},u_{\varepsilon t}\}$:
\begin{equation*}
\Omega =
\begin{bmatrix}
\varpi_Y & \varpi_{Yp}& 0 \\
\varpi_{pY}& \varpi_p & 0 \\
0 & 0 & \varpi_\varepsilon \\
\end{bmatrix}
\end{equation*}
\framebreak
% to be continued
\end{frame}



\subsection{Aggregate Implications}
\begin{frame}
\frametitle{}
\end{frame}

\section{Estimation}
\subsection{Method}
\begin{frame}
\frametitle{Estimation Steps}
\begin{itemize}
\item Step 1: Parameters for the joint process of aggregate income and prices(Appendix A)
\item Step 2: Parameters from the policy functions
\end{itemize}
\end{frame}

\begin{frame}
\frametitle{Estimation Strategy}
\begin{itemize}
\item Strategy: To find the parameters that bring data from the simulated model as close as possible to the data
\begin{itemize}
\item $\gamma$: matching three moments characterizing the cross sectional distribution as well as three moments characterizing the probability of scrapping a car(hazard function)
\item $\theta$: find the one to minimize the distance between the actual and simulated data 
\end{itemize}
\item Types of observations:
\begin{itemize}
\item Time series observations on sales, prices and income to match the sales predicted by our model
\item 
\end{itemize}
\end{itemize}
\end{frame}

\begin{frame}
\frametitle{Estimating $\theta$}
\begin{itemize}
\item Overall criterion:
$$
L(\theta)=\phi L^1(\theta)+L^2(\theta)
$$
\item First component: standard nonlinear least square criterion measuring the squared distance between observed and average predicted values of the variables
$$
L^1(\theta)
$$
\item Second piece: 
$$
L^2(\theta)
$$
\end{itemize}
\end{frame}

\subsection{Estimation Results}

\begin{frame}
\frametitle{Estimation Results(Table 2)}
% summarized
\begin{itemize}
\item Rate of depreciation of service flow($\gamma$): 34\% for France, 41\% for US; significant
\item Curvature estimates from nondurable consumption($\xi$): 1.7-1.8
\item Actual and predicted moments 
\item Probability of car breakdown($\delta$):1-2\%
\item $R^2$: % not understand
\item Over-identifying restrictions: % not understand
\end{itemize}
\end{frame}

\subsection{Time Series Representations}
\subsubsection{ARMA Representation}
\begin{frame}
\frametitle{ARMA Representation(Table 3}
\begin{itemize}
\item Methods:
\begin{itemize}
\item 
\end{itemize}
\item Results:
\begin{itemize}
\item
\end{itemize}
\end{itemize}
\end{frame}

\subsubsection{Impulse Response Functions}
\begin{frame}
\begin{itemize}
\item 
\end{itemize}
\end{frame}

\section{Decomposing the Results: What lies behind the Oscillations?}
\begin{frame}
\frametitle{Sources of the dynamics}
\begin{itemize}
\item A shock to income produces a dynamic in durable expenditures as agents respond differentially(i.e, agents with younger cars are less likely to respond to income variations than are agents with
older cars)
\item Dynamics induced by prices and income as these processes are serially correlated. Movements in these variables are represented by shifts in the probability of adjustment(hazard)
\end{itemize}
\end{frame}

\subsection{Sales}
\begin{frame}
\frametitle{Sales}
\end{frame}

\subsection{Decomposing the IRFs}
\begin{frame}
\frametitle{Decomposing the IRFs}
\end{frame}

\section{Robustness}
\begin{frame}
\frametitle{Robustness}
Restriction relaxed:
\begin{itemize}
\item Market for the sale of used cars
\item Borrow and lend
\end{itemize}
Methods:
\begin{itemize}
\item ARMA(1,1)
\item Impulse response functions from a linear VAR model.
\end{itemize}
\end{frame}

\subsection{Used Car Markets}
\begin{frame}
\frametitle{Used Car Markets}
\end{frame}

\subsection{Capital Markets}
\begin{frame}
\frametitle{Capital Markets}
\begin{itemize}
\item Cost of buying a durable good cannot be spread over time, thus implicity increasing the cost of such expenditures.
\end{itemize}
\end{frame}

\subsection{Robustness of Implied Dynamics of Car Sales}
\begin{frame}
\frametitle{Robustness of Implied Dynamics of Car Sales}
\begin{itemize}
\item
\end{itemize}
\end{frame}

\section{Conclusion}
% understand
\begin{frame}
\frametitle{Conclusion}
\begin{itemize}
\item Theme: aggregate time series implications of a model of consumption of both durables and nondurables at the household level
\item Model: Dynamic discrete choice, infrequent purchases of durables - impulse response functions
\item Contribution: 
\begin{itemize}
\item Solving the "durables puzzle" of Mankiw(1982)
\item Focus on the underlying parameters of the individual��s dynamic discrete choice problem; 
\item Emphasized properties of the cross sectional distribution of car ages; 
\item Time series implications that match certain features of the data
\end{itemize}
\item Decomposition: hazard function(most) and evolution of the cross sectional distribution(some)
\item Robustness: second hand markets; borrowing and landing
\item Further attention:
\begin{itemize}
\item Price endogenously
\item Household-level data
\end{itemize}
\end{itemize}
\end{frame}

\section{Appendix}
\subsection{Estimation Results for Joint Process of Income and Prices}
\begin{frame}
\frametitle{Estimation Results for Joint Process}
\begin{itemize}
\item
\end{itemize}
\end{frame}

\subsection{Extensions to Our Baseline Model}
\subsubsection{Used Car Markets}
\begin{frame}
\frametitle{Used Car Markets}
\begin{itemize}
\item
\end{itemize}
\end{frame}

\subsubsection{Capital Markets}
\begin{frame}
\frametitle{Capital Markets}
\begin{itemize}
\item
\end{itemize}
\end{frame}

\section{My Notes}
\subsection{Economic Notes}
\begin{frame}
\frametitle{Permanent Income Hypothesis(PIH)}
\begin{itemize}
\item Definition: a person's consumption at a point in time is determined not just by their current income but also by their expected income in future years��their "permanent income"
\item Permanent income: expected long-term average income.
\end{itemize}
\end{frame}

\subsection{Mathematical Notes}

\begin{frame}
\frametitle{Autoregressive(AR) Model\footnotemark}
AR(p):
$$
y_t = c + \sum_{i=1}^{p}\varphi_i y_{t-i}+\sigma v_t + \varepsilon_t
$$
\begin{itemize}
\item $\varphi_i$: parameters of the model
\item c: constant
\item $\varepsilon_t$: white noise
\end{itemize}
\footnotetext{https://en.wikipedia.org/wiki/Autoregressive_model}
\end{frame}

\begin{frame}
\frametitle{Moving-Average(MA) Model\footnotemark}
MA(q):
$$
X_t=\mu+\varepsilon_t+\theta_1\varepsilon_{t-1}+...+\theta_q\varepsilon_{t-q}
$$
\begin{itemize}
\item $\mu$: mean of the series
\item $\theta_i$: parameters of the model
\item $\varepsilon_{t-i}$: white noise
\end{itemize}
\footnotetext{https://en.wikipedia.org/wiki/Moving-average_model}
\end{frame}

\begin{frame}
\frametitle{Autoregressive-Moving-Average Model}
ARMA(p,q): the model with p autoregressive terms and q moving-average terms
$$
X_t = c + \varepsilon_t + \sum_{t=1}^pX_{t-i}+\sum_{i=1}^{q}\theta_i\varepsilon_{t-i}
$$
\begin{itemize}
\end{itemize}
\end{frame}

\begin{frame}
\frametitle{Vector Autoregression(VAR)}
VAR(p): evolution of a set of k endogenous variables over the same sample period
$$
y_t = c + \sum_{i=1}^p A_i y_{t-i}+e_t
$$
\begin{itemize}
\item $y_t$: k*1 vector
\end{itemize}
\end{frame}

\begin{frame}
\frametitle{VAR: structural vs. reduced form}
\begin{itemize}
\item Structural:
$$
B_0y_t=c_0+\sum_{i=1}^{p}B_i y_{t-i}+\varepsilon_t
$$
\item Reduced-form:
$$
y_t = c + \sum_{i=1}^p A_i y_{t-i}+e_t
$$
\end{itemize}
\end{frame}

\end{document}

